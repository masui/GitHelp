% compare current (README.md|index.html) with the file from 2 versions ago
% compare (README.md|index.html) with the one from (1|2|3|4) versions before
% compare current (README.md|index.html) with that from  ([0-9]+)-days ago

% 部屋を暗くする
% make the room darker 
% iot ......


\documentclass{sigchi}

% Use this section to set the ACM copyright statement (e.g. for
% preprints).  Consult the conference website for the camera-ready
% copyright statement.

% Copyright
\CopyrightYear{2018}
%\setcopyright{acmcopyright}
\setcopyright{acmlicensed}
%\setcopyright{rightsretained}
%\setcopyright{usgov}
%\setcopyright{usgovmixed}
%\setcopyright{cagov}
%\setcopyright{cagovmixed}
% DOI
\doi{http://dx.doi.org/10.475/123_4}
% ISBN
\isbn{123-4567-24-567/08/06}
%Conference
\conferenceinfo{UIST'18,}{May 07--12, 2016, San Jose, CA, USA}
%Price
\acmPrice{\$15.00}

% Use this command to override the default ACM copyright statement
% (e.g. for preprints).  Consult the conference website for the
% camera-ready copyright statement.

%% HOW TO OVERRIDE THE DEFAULT COPYRIGHT STRIP --
%% Please note you need to make sure the copy for your specific
%% license is used here!
% \toappear{
% Permission to make digital or hard copies of all or part of this work
% for personal or classroom use is granted without fee provided that
% copies are not made or distributed for profit or commercial advantage
% and that copies bear this notice and the full citation on the first
% page. Copyrights for components of this work owned by others than ACM
% must be honored. Abstracting with credit is permitted. To copy
% otherwise, or republish, to post on servers or to redistribute to
% lists, requires prior specific permission and/or a fee. Request
% permissions from \href{mailto:Permissions@acm.org}{Permissions@acm.org}. \\
% \emph{CHI '16},  May 07--12, 2016, San Jose, CA, USA \\
% ACM xxx-x-xxxx-xxxx-x/xx/xx\ldots \$15.00 \\
% DOI: \url{http://dx.doi.org/xx.xxxx/xxxxxxx.xxxxxxx}
% }

% Arabic page numbers for submission.  Remove this line to eliminate
% page numbers for the camera ready copy
% \pagenumbering{arabic}

% Load basic packages
\usepackage{balance}       % to better equalize the last page
% \usepackage{graphics}      % for EPS, load graphicx instead 
\usepackage{graphicx}      % for EPS, load graphicx instead 
\usepackage[T1]{fontenc}   % for umlauts and other diaeresis
\usepackage{txfonts}
\usepackage{mathptmx}
% \usepackage[pdflang={en-US},pdftex]{hyperref} なんかわからない
\usepackage{color}
\usepackage{booktabs}
\usepackage{textcomp}

% Some optional stuff you might like/need.
\usepackage{microtype}        % Improved Tracking and Kerning
% \usepackage[all]{hypcap}    % Fixes bug in hyperref caption linking
\usepackage{ccicons}          % Cite your images correctly!
% \usepackage[utf8]{inputenc} % for a UTF8 editor only

\usepackage{here} % [H]とするとその場所に配置されるらしい

% If you want to use todo notes, marginpars etc. during creation of
% your draft document, you have to enable the "chi_draft" option for
% the document class. To do this, change the very first line to:
% "\documentclass[chi_draft]{sigchi}". You can then place todo notes
% by using the "\todo{...}"  command. Make sure to disable the draft
% option again before submitting your final document.
\usepackage{todonotes}

% Paper metadata (use plain text, for PDF inclusion and later
% re-using, if desired).  Use \emtpyauthor when submitting for review
% so you remain anonymous.
\def\plaintitle{Bridging the Translation Gap with ExpandHelp}
\def\plainauthor{First Author, Second Author, Third Author,
  Fourth Author, Fifth Author, Sixth Author}
\def\emptyauthor{}
\def\plainkeywords{Authors' choice; of terms; separated; by
  semicolons; include commas, within terms only; required.}
\def\plaingeneralterms{Documentation, Standardization}

% llt: Define a global style for URLs, rather that the default one
\makeatletter
\def\url@leostyle{%
  \@ifundefined{selectfont}{
    \def\UrlFont{\sf}
  }{
    \def\UrlFont{\small\bf\ttfamily}
  }}
\makeatother
\urlstyle{leo}

% To make various LaTeX processors do the right thing with page size.
\def\pprw{8.5in}
\def\pprh{11in}
\special{papersize=\pprw,\pprh}
\setlength{\paperwidth}{\pprw}
\setlength{\paperheight}{\pprh}
\setlength{\pdfpagewidth}{\pprw}
\setlength{\pdfpageheight}{\pprh}

% Make sure hyperref comes last of your loaded packages, to give it a
% fighting chance of not being over-written, since its job is to
% redefine many LaTeX commands.
\definecolor{linkColor}{RGB}{6,125,233}
% \hypersetup{%
%   pdftitle={\plaintitle},
% % Use \plainauthor for final version.
% %  pdfauthor={\plainauthor},
%   pdfauthor={\emptyauthor},
%   pdfkeywords={\plainkeywords},
%   pdfdisplaydoctitle=true, % For Accessibility
%   bookmarksnumbered,
%   pdfstartview={FitH},
%   colorlinks,
%   citecolor=black,
%   filecolor=black,
%   linkcolor=black,
%   urlcolor=linkColor,
%   breaklinks=true,
%   hypertexnames=false
% }

% create a shortcut to typeset table headings
% \newcommand\tabhead[1]{\small\textbf{#1}}

% End of preamble. Here it comes the document.


\def\GH{\textsf{GitHelp}}
\def\SB{\textsf{Scrapbox}}
\def\EH{\textsf{ExpandHelp}}
\long\def\tt#1{\texttt{#1}}
\long\def\stt#1{\texttt{\fontsize{9pt}{0pt}\selectfont{#1}}}
\long\def\sf#1{\textsf{#1}}
\long\def\ssf#1{\textsf{\fontsize{9pt}{0pt}\selectfont{#1}}}
\long\def\qtt#1{``\tt{#1}''}   % quote tt
\long\def\sqtt#1{``\stt{#1}''} % smal quote tt
\long\def\qsf#1{``\sf{#1}''}
\long\def\sqsf#1{``\ssf{#1}''}
\long\def\qit#1{``\textit{#1}''}

\begin{document}

\title{\plaintitle}

\numberofauthors{3}
\author{%
  \alignauthor{Toshiyuki Masui
    \affaddr{Keio University}\\
    \affaddr{Fujisawa, Japan}\\
    \email{masui@pitecan.com}}\\
  \alignauthor{Jun Kato\\
    \affaddr{AIST}\\
    \affaddr{Tsukuba, Japan}\\
    \email{junkato}}\\
}

\maketitle

\begin{abstract}
  We introduce a flexible command translation system that can generate
  a complex command string from vague keywords
  given by the user.
  %
  % Although intuitive GUI are 
  % command-line interface (CLI) is stil widely used everywhere,
  %
  When people use computers to perform tasks,
  there is usually a huge mismatch between the users' intention
  and the required action.
  % the language and vocabularies used for the task are completely different.
  %
  For example, when a Chinese person wants to ``房間暗一点'' (make the room darker),
  % he may have to translate his intention into an English expression ``make the room darker'',
  he may have to find that he should ``turn off the ceiling light'' for the task,
  translate it to a required action ``toggle the wall switch''
  or a command like ``\tt{\$ iot clight 0}''.
  %
  % When a user wants to ``make the room darker'',
  % he should translate it to a real action like ``turn off the ceiling light''.
  % To do so,
  % he may have to ``toggle the wall switch''
  % or issue a command like ``\tt{\$ iot ceilinglight 0}''.
  % If the user is a Chinese speaker, he might have to translate his intention
  % ``房間暗一点'' into ``make the room darker'',
  % ``turn off the ceiling light'', 
  % ``toggle the wall switch'',
  % and finally to a command like ``\tt{\$ iot ceilinglight 0}''.
  %
  There is a huge gap between ``房間暗'' and ``\tt{\$ iot clight 0}'', and
  people are perpetually suffering from such multi-level ``\textit{translation gaps}''.
  Translation gaps exist even for experienced computer users, but
  they are serious for vast amount of people around the world.
  %
  We propose a simple and general translation framework
  {\EH} that can be used in various situations
  where such translation is required.
  We show how our framework can be used
  in the {\GH} system
  with which user can easily generate complex \tt{git} commands
  from fragments of users' intentions.
\end{abstract}

\category{H.5.m.}{Information Interfaces and Presentation
  (e.g. HCI)}{Miscellaneous} \category{See
  \url{http://acm.org/about/class/1998/} for the full list of ACM
  classifiers. This section is required.}{}{}

\keywords{\plainkeywords}

\section{Introduction}

When people want to perform a task using modern artifacts,
people frequently find difficulty handling them properly.
%
Even when manuals and help systems are provided,
people don't use them\cite{Novick:2006:WDP:1166324.1166329}
partly because of the vocabulary problem\cite{Furnas:1987:VPH:32206.32212}.
%
% Twitterでのアンケート: 80\% が全然使わない
% https://gyazo.com/2dd525388cce34c916638f51bad386c7
% 
% やりたいと思ったことがすぐできればどれほど嬉しいことか
% そもそも検索や翻訳が必要なものであるならそれを楽にすればよい
%
We should translate our vague intentions in our brain into actions required for the artifacts.
When we want to watch a movie, we may pick up a DVD disk, put it into a
DVD player, push the play button, and set up the audio and the monitor.
When we feel thirsty, we may go to the kitchen, pick up a glass, and
turn the faucet on to fill the glass with water.
All the living creatures are performing such translations
between intentions and actions without difficulty, but
in the modern world, people are suffering from
serious gaps between what they want to do and what they have to do.

Even when people vaguely know what they have to do for what they want to do,
it is often not easy for them to perform the task properly.
Even when people know that they can watch a movie on the Internet,
finding a movie and playing it is a difficult task for people without experiences.
It would be nicer if they could watch a movie on the net
just by showing a vague title or attributes of the movie
in various languages.

% We can use search engines to find clues for performing the tasks, but
% it is not always easy for people in the world, since they may have
% to use different languages to look for the information.

We propose a simple, flexible and powerful
translation framework with which users can generate a complete
command string from fragments of users' vague intentions.
%
We provide a database consisting of pairs of
the description of the task that users want to perform (e.g. 房間暗一点)
and
the actual command string required by the system (e.g. \tt{\$ iot clight 0}).
Descriptions can be written in any natural language using regular expressions
and the database on the Wiki pages can be edited by any user
who wants to improve the database.

% やりたいことの[* 平易な表現と実行コマンドの組を用意]して[* 共有]しておいて、[* 簡単に検索/実行]できるようにする
% 表現は[* 何語で書いてあってもかまわない]
% 例えば映画作品見たいならAmazon PrimeとNetflix両方に作品があったりする
% コマンドの例でいえば何をどの順番でパイプするかいろいろ選択肢がありうる
% 全部出てくれると便利ですね(自分の知らなかったコマンド記法が学べたりする)
% いろんな候補は出ます [増井俊之.icon]
% 「Amazonでゴジラを見る」「Netflixでゴジラを見る」みたいに
% コマンドの勉強になる とは思っています [増井俊之.icon]
% そういう話は記述した方がいいかも
% `@{2 days ago}`なんて記法を知ってる人は見たことなし [増井俊之.icon]
% 今は自力で打てるようになった!
% これを[* IMEみたいなインタフェースで利用する]
% ちなIMEはアジアでは常識ね
% [* データはWikiで共有しておいて誰でも追加/修正できる]
% 自分のやり方を定義してもいい

Contributions of this paper are as follows:

\begin{enumerate}
% 各種技術を組み合わせた翻訳支援というパラダイムの提案
\item Introduction of ``\textit{translation paradigm}'' for
  generating a correct instruction to an artifact from
  the user's vague intention in natural language.
\item General-purpose flexible dynamic search system based on regular expressions.
% そのための柔軟なインフラ (ExpandHelp)
% \item IME-like command etnry interface
% やりたいことを実行するインタフェース (IME的なもの)
\item Data sharing infrastructure for the above strategy.
% 共有/編集しやすい正規表現DSLとそのインプリ(Scrapbox+)
% \item これが唯一の解決法ではないが、こういう考え方がポピュラーになって欲しい
\end{enumerate}

\section{Example: GitHelp}

We implemented the {\GH} system
with which a user can translate his vague intention in various languages
into a complete \tt{git} command
using the {\EH} framework.

\subsection{Entering a command}

We assume that the user uses \tt{bash} for his software
development activities.

% \begin{figure}[h]
%   \includegraphics[width=8cm,bb=0 0 670 254]{figures/bash1.png}
%   \caption{Bash shell.}
%   \label{bash1}
% \end{figure}

When a user wants to compare the latest \tt{README.md} file
with the same file from 2 versions before,
he might try to use \tt{git} with an argument
``\verb|HEAD^^|''
for specifying the old version.

\begin{quotation}
  \verb|$ git diff HEAD^^ README.md|
  \par
  \verb|$|
\end{quotation}

However, nothing happens here because
\verb|HEAD^^|
specifies a file included in the older commit,
and the command above only compares the \stt{README.md}
with the file included in an old commit, where
\stt{README.md} might be the same as the latest version.

If {\GH} is installed,
the user can type a special shortcut key
% (e.g. \tt{Cmd-Ctrl-G})
to invoke {\GH} after typing \stt{git RE 2} to see how he can
perform the task.

\begin{figure}[h]
  \includegraphics[width=9cm,bb=0 0 1000 600]{figures/githelp_e2.png}
  \caption{Invoking {\GH} after typing \stt{git rea 2}.}
  \label{bash1}
\end{figure}

Here, various candidate entries with
descriptios and command strings are listed.
One of the entries says
``\ssf{Compare} \stt{README.md} \ssf{with the one from 2 versions before}''.
If the user thinks that that's what he wants to do,
he can select the entry by typing arrow keys and type the enter key.
Then the user's input string is replaced by the right command string
that satisfies the user's need.

A description like ``2 days ago'' is also shown in the candidate list.
This means that the user can use the same string \sqtt{git RE 2}
for getting the \stt{README.md} file of the day before yesterday.
The command,
``{\fontsize{9pt}{0pt}\selectfont\verb|$ git diff '@{2 days ago}' README.md|}'',
may not be familiar to most \tt{git} users,
but users can perform this task only by giving \sqtt{RE} and \sqtt{2}.

A Japanese \stt{git} user can do the same thing just like
English-speaking users.
He can user the term \qtt{比}(compare) to find how he can perfor the job.

\begin{figure}[H]
  % \includegraphics[width=10cm,bb=0 0 1290 866]{figures/githelp1.png}
  \includegraphics[width=12cm,bb=-100 -100 1190 766]{figures/githelp1.png}
  \caption{Invoking {\GH}.}
  \label{bash2}
\end{figure}

If the user wants to know more about the command line,
he can click the
\raisebox{-2pt}{\includegraphics[height=11pt,bb=0 0 200 200]{figures/info-xxl.png}}
button and see the Wiki page shown in Figure \ref{scrapboxpage}.
The user can edit the page if he wants to add information or finds an error.

\begin{figure}[htb]
\begin{verbatim}
\end{verbatim}
% \centerline{\includegraphics[width=100mm,bb=0 0 1428 1288]{figures/scrapbox1.png}}
\centerline{\includegraphics[width=100mm,bb=-50 -50 1000 900]{figures/scrapbox1.png}}
\caption{A {\SB} page for \tt{git diff}.}
\label{scrapboxpage}
\end{figure}

\section{Implementation}

% {\EH} search algorithm and database structure

\subsection{ExpandHelp}

A flexible information retrieval mechanism ``{\EH}'' is used in {\GH}.

Help data used in {\EH} are provided as a set of data entries
consisting of a regular expression that represents the help entry description
and a corresponding command pattern.
A description string is used to produce natural language texts
that are easily understandable by users,
while command patterns are used for invoking commands or calling library functions
corresponding to the description.
%
For example, a help entry for showing the difference between
current file and older file can be like this:

\begin{quote}
  \textbf{description}: \\
  \stt{Compare (\#\{files\}) with the one from (1|2|3|4) versions before}\\
  \textbf{command}: \\
  {\fontsize{9pt}{0pt}\selectfont\verb|git diff $(git rev-list -n #{$2} HEAD -- #{$1})^ -- #{$1}|}
\end{quote}

The first regular expression, the description part,
is for generating texts like
\sqsf{Compare README.md with the one from 1 versions before}, 
\sqsf{Compare index.html with the one from 2 versions before}, etc.,
and the second part, command part,
is used to generate a command to execute a \tt{git} command.
In the first example, \sqsf{README.md} and \sqsf{1}
match the first and second parentheses of the regular expression, and
these values are assigned to \tt{\$1} and \tt{\$2}
for the command part, generating a command
``{\fontsize{9pt}{0pt}\selectfont\verb|$ git diff $(git rev-list -n #{$2} HEAD -- #{$1})^ -- #{$1}|}''.

The description part can be written in any natural language.
For example,

\begin{quote}
  \stt{(\#\{files\})を(1|2|3|4)個前のものと比較する}
\end{quote}
  
can be used for Japanese users.
In this case,
\sqsf{README.mdを1個前のものと比較する},
\sqsf{README.mdを2個前のものと比較する}, etc.
are used for theh matching.

% When the matched pattern like \qsf{San Francisco}
% cannot be directly used in the command,
% parameters for the command can be put after the description string,
% separated by a TAB character.
% For example, when a number like ``\tt{29247}'' should be used
% for checking the
% weather of San Francisco, we can describe the help entry like this:
% 
% \begin{quote}\small
%  description: \\
%  \tt{Check the weather of (San Francisco{\textbackslash}t29247| New York{\textbackslash}t23164)} \\
%  command: \\
%  \tt{open 'http://local.msn.com/ten-day.aspx?\\eid=\$1'}
% \end{quote}
%
% Here, selecting a menu entry \qsf{Check the weather of San Francisco}
% will produce a command
% \tt{open 'http:{\slash}{\slash}local. msn.com{\slash}ten-day.aspx?eid=29247'}.

%% \subsection{(section written for GitHelp)}
%% 
%% {\EH} is a general-purpose flexible help generation system
%% that has the following characteristics.
%% 
%% \begin{itemize}
%% \item Use description-execution pairs for the translation
%% \item Standard regular expression is used for the specification of the description
%% \item The specification RegExp is expanded and filtered by the user's input string
%%   real-time
%% \end{itemize}
%% 
%% An example of the desc-exec pair can be the following:
%% 
%% \begin{verbatim}
%%   (delete|remove) a (#{file})
%%   /bin/rm $2
%% \end{verbatim}
%% 
%% ``remove \verb|README.md|''
%% into
%% ``\verb|rm README.md|
%% 
%% First, the spec like \stt{\#\{file\}} is expanded to a list of filenames like
%% \verb+README.md|Makefile|test.c+,
%% and the spec string becomes
%% ``\verb+(delete|remove) a file (README.md|Makefile|test.c)+''.
%% Then this regular expression is expanded to all the possible string that match this regexp like
%% \begin{verbatim}
%% delete a file README.md
%% remove a file README.md
%% delete a file Makefile
%% remove a file Makefile
%% ...
%% \end{verbatim}
%% 
%% And then all these strings are compared with the user's specification like
%% ``\verb|del REA|'',
%% and if a matche is found,
%% this is translated to
%% ``\verb|delete file README.md|''.
%% 
%% If ``\verb|destroy|'' should also be used for the spec,
%% the spec can be modified to 
%% \stt{(delete|remove) a (\#{file})}.
%% 
%% Any kind of specs in any language can be used for the specification.
%% For example,
%% 
%% \begin{verbatim}
%%   \#{file}を(消す|消去する)
%%   /bin/rm $1
%% \end{verbatim}
%% 
%% can be  used for Japanese users, who might want to delete
%% one of the files by specifying ``消す''.
%% (消す means 'delete' in Japanese).

\subsubsection{Generating help menu entries}

Finding appropriate entries from the huge help document space is performed in two steps.
First, we parse the regular expression to
generate a state transition diagram that represents all the strings represented
in the regular expression.
Second, we generate all the description strings by expanding the regular expression
into a tree, and filter the generated strings by the pattern specified by the user.
Efficient pattern matching is performed every time a new description text is generated,
and only the best-matched descriptions are shown to the user as menu entries.
We call this the ``\textit{generate-and-filter}'' technique,
and we will describe the details later.

\paragraph{Phase 1: Generating a state transition diagram from a regular expression}

Regular expressions are widely used for finding patterns in text strings
in modern programming languages and in the Unix environment.
In {\EH}, we use regular expressions for
generating various patterns of descriptions in a short form.
For example, we use a short regular expression
\sqtt{(remove|delete|erase) (data|file)}
for representing expressions like
\sqsf{remove data},\sqsf{erase file}, etc.

Converting a regular expression to a state transition machine is a straightforward task.
When we have a regular expression
\sqtt{Compare (README.md|index.html|package.json) with the one (1|2|...|10) versions before}, 
we can parse the string and convert it into a state transition diagram
shown in Figure \ref{statemachine1}.

\begin{figure}[htb]
\includegraphics[width=90mm,bb=0 0 571 126]{figures/statemachine.pdf}
\caption{State transition diagram for finding a restaurant.}
\label{statemachine1}
\end{figure}

By traversing the nodes and links in this state machine,
we can generate the following description texts.

\begin{quote}
\small
\ssf{Compare README.md with the one 1 versions before} \\
\ssf{Compare index.html with the one 1 versions before} \\
...\\
\ssf{Compare package.json with the one 10 versions beforef}\\
\end{quote}

\paragraph{Phase 2: Generating texts from a state transition diagram}

Using the state transition diagram shown in Figure \ref{statemachine1},
{\EH} generates all the strings represented in the regular expressions,
and filter them by the pattern provided by the user.

We can generate texts from a state transition diagram by traversing nodes one by one.
Starting at the start node
(\raisebox{-2pt}{\includegraphics[height=10pt,bb=0 0 40 40]{figures/startnode.pdf}})
in Figure \ref{statemachine1},
we can visit other nodes via edges and generate a tree of generated texts
shown in Figure \ref{gentree1}.
After visiting the initial node
\raisebox{-2pt}{\includegraphics[height=10pt,bb=0 0 40 40]{figures/startnode.pdf}},
the system generates a string \sqsf{Compare} and proceeds to the next node.
In the second generation,
the system can add \sqsf{README.md}, \sqsf{index.html}, and \sqsf{package.json}
to \sqsf{Compare}, generating
\sqsf{Compare README.md}, \sqsf{Compare index.html}, etc.
The system can repeat finding edges from previously visited nodes,
and eventually generates all the strings described above.
Of course, it is impossible to generate all the strings
from a regular expression like ``\tt{(0|1)+}'' that represents infinite length of
strings consisting of \tt{0}s and \tt{1}s, so the generation should be
terminated at certain generation.

\begin{figure}[htb]
\includegraphics[width=85mm,bb=0 0 643 398]{figures/gentree1.pdf}
\caption{Generating a text tree from a state transition machine.}
\label{gentree1}
\end{figure}

\subsubsection{Generate-and-Filter}

Generating all the texts in this way before finding matched strings is
inefficient, because the amount of generated texts can easily become huge.
Instead, the system performs the patter matching as soon as a text is generated,
for saving time and memory.

``Generate-and-test''\footnote{
  {\sf http:{\slash}{\slash}en.wikipedia.org{\slash}wiki{\slash}Generate\_and\_test}
}
is a simple and effective strategy
used for solving puzzles and AI problems.
For example,
the ``8-Queen'' problem can be solved simply by
generating all the possible queen layouts and checking if
two or more queens are laid out on the same row, column, or diagonal line.
Although this strategy is simple, the cost of
generating all the possible solutions is prohibitive, since
the solution space of the 8-Queen puzzle is $8^8 = 16,777,216$,
which is not a small amount even for today's computers.

To solve this problem,
controlling the generation part from the testing part is effective.
In the 8-Queen problem,
whenever the testing part finds two queens in the same row or column,
it can tell the generation part to
give up current layout in the early stage and proceed to the next layout.

Similarly in our case,
it is not efficient to perform the matching operation
after generating all the texts from the regular expression,
and it is better to calculate the matching
at the time of generating each text.
We call it the \qit{generate-and-filter} strategy,
and implemented the algorithm in {\EH}.
Unlike simple puzzle problems where
conditions are strict and finding one solution is enough,
our goal is to find help description strings
that fits the query pattern while allowing errors.
For this goal,
the system should perform approximate pattern matching
without sacrificing processing speed.
%
We could implement flexible generate-and-filter by using a simple and efficient
approximate pattern matching algorithm based on the ``shifter algorithm''.

\subsubsection{Approximate pattern matching by shifter algorithm}

The ``shifter algorithm''\cite{Wu:1992:FTS:135239.135244}
is a simple and efficient
text search algorithm which has interesting features
not found in more common pattern matching algorithms like
KMP\cite{KMP}, Boyer-Moor\cite{Boyer:1977:FSS:359842.359859}, etc.

When a user wants to find a word \qsf{README} in a text,
he can use a patter matching state machine like below,
where the gray circle denotes that the state is active.

\begin{figure}[h]
\includegraphics[width=70mm,bb=0 0 439 73]{figures/readme1.pdf}
\caption{A state machine for \qsf{README}.}
\label{readme1}
\end{figure}

Initially, only the leftmost node is active, but when the
state machine receives \qsf{R}, both the first and the second node become active
and the activation state will change to the following pattern.

\begin{figure}[h]
\includegraphics[width=70mm,bb=0 0 439 73]{figures/readme2.pdf}
\caption{After receiving \qsf{R}.}
\label{readme2}
\end{figure}

When \qsf{README} is given to the state machine,
the rightmost node becomes active.

\begin{figure}[h]
\includegraphics[width=70mm,bb=0 0 439 73]{figures/readme3.pdf}
\caption{After receiving \qsf{README}.}
\label{readme3}
\end{figure}

Although this state machine is a nondeterministic finite automata (NFA),
only 10 bits are required to represent the active/inactive states of the nodes,
meaning that the whole states can be represented by a single integer value.
Also,
the state transition can be calculated by a simple combination of
logic and bit shift operations.

%% We will describe how the pattern matching can be performed using a C code.
%% 
%% First, a variable for current state is
%% set to \stt{0x80000000}, representing the initial pattern
%% shown in Figure \ref{readme1}.
%% 
%% {
%% \scriptsize
%% \begin{verbatim}
%%  #define INITPAT 0x80000000
%%  int state = INITPAT;
%% \end{verbatim}
%% }
%% 
%% When character \qsf{r} is received, the MSB will be shifted to the right
%% by one bit, and logical OR is calculated with the initial value, \stt{0x80000000}.
%% Since the 7th node also accepts \qsf{r}, it should also be shifted.
%% The state transition can be described like below.
%% 
%% {
%% \scriptsize
%% \begin{verbatim}
%%  mask['r'] = 0x82;
%%  state = INITPAT | ((state & mask['r']) >> 1);
%% \end{verbatim}
%% }
%% 
%% The shift ``mask'' value can be easily calculated from the given
%% pattern string beforehand.
%% 
%% {
%% \scriptsize
%% \begin{verbatim}
%%   int maskpat;
%%   init(pat){
%%     maskpat= INITPAT;
%%     for(char *s=pat;*s;s++){
%%       mask[*s] |= maskpat;
%%       maskpat >>= 1;
%%     }
%%     acceptpat = maskpat;
%%   }
%% \end{verbatim}
%% }
%% 
%% The pattern matching function can be described like below.
%% 
%% {
%% \scriptsize
%% \begin{verbatim}
%%  match(s){
%%    int state = INITPAT;
%%    for(s=text;*s;s++){
%%      state = INITPAT | ((state & mask[*s]) >> 1);
%%    }
%%    return ((state & acceptpat) != 0);
%%  }
%% \end{verbatim}
%% }

\subsubsection{Approximate pattern matching using the shifter algorithm}

The state machine shown in Figure \ref{readme1} can accept only one pattern
(\sqsf{README}), but
it can be easily expanded to detect texts which contain a word
similar to the pattern (e.g. \sqsf{READY}).

%   *  ** 
%  risto rante
%  restaurant
\begin{figure}[htb]
\begin{verbatim}
\end{verbatim}
\centerline{\includegraphics[width=40mm,bb=0 0 104 51]{figures/readme-mismatch.pdf}}
\caption{Matching error between \qsf{README} and \qsf{REDDY}.}
\label{readme-reddy-mismatch}
\end{figure}

Adding three more states to Figure \ref{readme1}, we can perform more
sophisticated pattern matcher which can accept strings with
0 to 3 matching errors.

\begin{figure}[htb]
\includegraphics[width=85mm,bb=0 0 443 282]{figures/readme-ambig.pdf}
\caption{A pattern matcher accepting 0-3 errors.}
\label{shifterambig}
\end{figure}

Figure \ref{shifterambig} shows a state machine which can detect a string
which matches \sqsf{README} with 0 to 3 mismatches.
Each additional row of nodes represents a matcher with one error,
two errors, and three errors, respectively.
Vertical and diagonal transition edges are added to allow
transitions based on spelling errors.

Initially, only the bottom-left node is active.
When a character other than \sqsf{R} is detected, 
the transition labeled \sqsf{*} (wildcard) is activated,
and connected nodes become active.
At the same time, links labeled as ``$\epsilon$''
is also activated without any input character.
With these additional links, this expanded machine can detect a text which
matches \sqsf{README} with 0 to 3 errors.

% \begin{figure}[htb]
% \includegraphics[width=82mm]{figures/restaurant-ristrante.pdf}
% \caption{Accepting ``\tsf{ristorante}'' using a matcher for ``\tsf{restaurant}''.}
% \label{restaurant-ristorante}
% \end{figure}

The transition of active nodes while reading
\sqsf{REDDY} is shown in Figure \ref{restaurant-ristorante}.
After reading \sqsf{REDDY},
only the upper-right node becomes active,
denoting that \sqsf{REDDY} matches
\sqsf{README}'' with 3 matching errors.

The state transition shown in 
Figure \ref{restaurant-ristorante} looks complicated, but
the matching state can be represented by only four integer variables, and
the matching algorithm is not very different from the algorithm
shown above.

\subsubsection{Using the matcher for generate-and-filter}

We can use this approximate pattern matcher while generating
help texts by traversing the state transition machine.
%
Every time a new node is generated by traversing an edge,
matching state is calculated and stored in the generated node.
The status pattern is calculated from the status pattern
stored in the preceding node and the string associated with the edge.

% \begin{figure}[htb]
% \includegraphics[width=85mm]{figures/ristfrans1.pdf}
% \caption{A pattern matcher for ``\ttt{rist frans}''.}
% \label{ristfrans1}
% \end{figure}
% 
% \begin{figure*}[bht]
% \centerline{\includegraphics[width=110mm]{figures/gentree2.pdf}}
% \caption{Matching statuses for ``\ttt{rist frans}'' associated with text generation nodes.}
% \label{gentree2}
% \end{figure*}

% Figure \ref{ristfrans1} shows the matcher for \qtt{rist frans}\footnote{
%   A space (\qtt{ }) in the input is treated as a wildcard character.
% },
% and Figure \ref{gentree2} shows how the matching status is calculated while the
% description strings are generated from the state transition machine
% shown in Figure \ref{statemachine1}.
% The matching status at each node is calculated from the matching status
% stored in the parent node and the string associated with the edge.
% For example, when the system generates the string
% \qsf{Find a Chinese restaurant in San Francisco on Yelp.com},
% the matching status is calculated from the status data at the previous node at
% \qsf{Find a Chinese restaurant in San Francisco}, and
% the system can tell that it matches \qtt{rist frans} with two matching errors,
% as soon as the string is generated.

% The system is keeping the list of generated strings
% which match the pattern with zero to three errors
% and it displays the list with minimal number of errors
% when generating menu entries.
% 
% Since the pattern matching operation is performed at the time of text generation,
% the whole generate-and-filter calculation is quick.
% Although current version of ExpandHelp is implemented in MacRuby,
% the user can always get the result within 1 second.
% A portion of the help data in Ruby is shown in Figure \ref{helpdata}.

% \begin{figure*}[bt]
% \centerline{\includegraphics[width=146mm]{figures/04ad7dee5679450e5390821745b5a0b7.png}}
% \caption{A portion of the help data in Ruby.}
% \label{helpdata}
% \end{figure*}

\subsection{Sharing the database on the Web}

The data used in {\GH} are represented as ``{\SB}'' documents.
%
{\SB} is a general-purpose real-time Wiki system for data sharing.
{\SB} users can edit the contents of Wiki pages directly
on Web browsers just like text editors,
using simple markup symbols.

The data used for {\GH} is shown in Figure \ref{scrapboxpage}.

% \begin{figure}[htb]
% \begin{verbatim}
% \end{verbatim}
% % \centerline{\includegraphics[width=100mm,bb=0 0 1428 1288]{figures/scrapbox1.png}}
% \centerline{\includegraphics[width=100mm,bb=-50 -50 1000 900]{figures/scrapbox1.png}}
% \caption{A {\SB} page for \tt{git diff}.}
% \label{xxxxx}
% \end{figure}

This page describes the usage of \sqtt{git diff}, and
the page contains command specifications in {\EH} format
in addition to the description of \sqtt{git diff}.
%
% Specs and command lines are specified with a special symbol, and
% all other documents can be written just like standard manuals and help documents.

Although usage examples are often shown on Unix man pages,
they cannot be used as the database of online help system.
On the other hand,
a {\SB} document contains both the manual page and the help database
at the same place.
%
This strategy is similar to the
``Literate Programming''\footnote{
  \ssf{https:{\slash}{\slash}en.wikipedia.org{\slash}wiki{\slash}Literate\_programming}
} style,
where source codes and documents are put in the same document.

Also, since the database is on the Wiki page, any member of the
Wiki can correct the document or add additional {\GH} specifications.
When a user could not get a good support from {\GH}, he can add
{\EH} specifications to the {\SB} document himself
so that he and other people can get an appropriate support next time.

\section{Related Work}

\subsection{Intelligent help systems}

There have been many researches on intelligent help systems\cite{Delisle:2002:UIH:606412.606415},
but few number of projects seem to be going on these days,
maybe because now we can use the Internet for getting intelligent help from
real people,
either by searching Web pages or 
by asking in user forums like StackOverflow\footnote{stackoverflow}.
Information on the net is rich these days, and we can easily
find somebody who can answer questions intelligently.
This is a good news, but 
asking many trivial questions on the net is not considered to be a
good manner, and
using a powerful and flexible local help system like {\EH} is
preferable before asking other people.

\subsection{Input Methods}

Text input systems for non-English languages are widely used
in the world,
and they are called as ``Input Methods'' (IMs).
%
Many research and development on IMs have been going on for years, and
people are using them every day for entering texts in their mother language.

\begin{figure}[h]
  \includegraphics[width=8cm,bb=0 0 976 670]{figures/nyuuryoku-ime.png}
  \caption{An IM for Japanese text input.}
  \label{bash1}
\end{figure}

IMs are one form of translation systems that convert one
expression (pronunciation, character shape, etc.)
into a textual form of a language.
In the above example, the pronunciation ``kanji''
is used for getting ``漢字''.

IMs are very popular in the world these days,
and it is very common for people to 
use translation systems for using computers.
Using an additional translation system like {\GH} is ...

\subsection{Text and code completion}

On text editors and Unix shells,
``text completion'' has been available for a long time.
%
When a user types the TAB key after typing \sqtt{la} on the Unix shell,
\sqtt{latex}, \sqtt{latex}, and other commands are shown as candidates.
When a user types \sqtt{ls RE} and types the TAB key,
Unix shell checks the directory, find the \stt{README.md} file, and
replace the user's input with \sqtt{ls README.md}.
This behavior is called ``text completion'', and many variations of
this idea have been proposed and adopted in text editors and command line editors.

User's history data can also be used for the completion function.
The Reactive Keyboard\cite{ReactiveKeyboard}
accumlates the user's command history and use the data
for predicting the user's next input.

% Programing-by-Example (PBE) systems\cite{Cypher}\cite{Lieberman}.

Gitsome\footnote{\textsf{https:{\slash}{\slash}github.com{\slash}donnemartin{\slash}gitsome}}
helps \tt{git} users by
dynamically showing possible arguments and related manuals on the Unix shell,
just like {\GH}.
The goal of Gitsome is alomost the same as {\GH} and we agree with its concept,
but Gitsome does not support ...

\subsection{Smart IDEs}

Many smart IDEs for suggesting codes have been proposed resently.

% 賢い補完
Little's system\cite{Little:2006:TKC:1166253.1166275}
generates a complete JavaScript code snippet from keyword fragments.
For example,
\sqtt{ActiveDocument .PageSetup .LeftMargin = InchesToPoints(2)}
can be generated from keywords like
\sqsf{left},
\sqsf{margin},
\sqsf{4}, and
\sqsf{inches}.
The system generates the code using a template database and heuristics
for the target system.
It is useful for a special environment, but
users of the system cannot modify the algorithm or the database
even when appropriate suggestions cannot be give by the system.

% キーワードの羅列からコマンドを作成するという考えはとても正しいと思うが、実装がいかがなものか?
% キーワードぐらいは思い出せるプログラマが対象になっている
% 「margin」とか「left」とかいう単語は思い出す必要がある
% 「ちょっと字下げ」とか言っても駄目
% キーワードが違ってると駄目?
% いろいろズルをしてるようだ
% [[ActiveDocument]]はよく出てくるので特別扱いしているとか
% ステミングをしている
% ``the'' ``to'' などは捨ててるのだろう
% そもそもリカーシブな検索アルゴリズムにかなり無理がありそう
% テンプレートを作るのは大変ではないのか?

% https://www.youtube.com/watch?v=93vZAmLyOQY
AnyCode\cite{Gvero:2015:SJE:2814270.2814295} is another system
that generates a complete code snippet from the user's
input in natural language.
AnyCode can handle synonyms like ``make'' == ``create'',
曖昧検索はできず、かなりキーワードを指定する必要はある

% [anyCode]というシステム。自然言語キーワードを入力するとコードスニペットが表示される。
% [[copy fileA fileB]]みたいなキーワードから[[FileUtil.copyFile(new File(fileA), new File(fileB))]]みたいなコード候補を生成する
% [自然言語処理]を行なっている
% [https://github.com/tihomirg/nlpcoder/tree/noola GitHub]
% 曖昧検索はできない
% 真面目に自然言語を入れる必要がある
% `123`みたいなパラメタは使えるか?
% 使えると思うが例には出てこない
% コードの説明文は出ない?
% FileUtil.copyFile(new File(fileA), new File(fileB)) と言われても、どちらが新しいファイルなのかわからない

Active Code Completion\cite{Omar:2012:ACC:2337223.2337324}
is another code completion system for the Eclipse IDE.
A database called \textit{palette} are used for code completion and coding support
for Java classes.
When a user wants to write a code for drawing a rectangle in purple,
the user should just tell ``purple'' to the system,
and the system shows the user a color selection window
as well as the Java code for drawing a purple rectangle.
This is a useful feature for the user to write a code including selecting
a color, but paletts are difficult to create, and should be provided
as IDE plugins.
% [/ palette] と呼ばれるテンプレートを用意して[IDE]でのコード補完をサポートする
% `getDefaultColor()` を定義したいとき`purple`と入力すると、それに対応した候補が提示されるとか
% 普段使ってる環境で有益な候補が表示されると嬉しいだろうと言っている
% [[コメント]]
% [Jun Kato.icon]
% > Active code completionも思い出しました。型ごとに適した入力インタフェースを出すコード補完インタフェースのnn提案です。確か。
% [増井俊之.icon]
% IDEでどういう機能が欲しいかサーベイして設計したことになっている
% 色選択と正規表現のためのコード生成をサポートした例が紹介されている
% 一般的な補完コマンド起動によって呼び出されるようだ
% サポートウィンドウはHTML5で生成する
% しかし両者とも[/palette]の作成はかなり大変である
% 普通のユーザが作れるようなものではない
% [* 作成がすごく大変]で、[* 特定のIDEでしか使えない]という問題があると思う

Han's Abbreviation Completion system\cite{Han:2009:CCA:1747491.1747530}
can generate a code snippet from abbreviation strings give by the user.
For example, a snippet like \sqtt{chooser. showOpenDialog()}
can be generated from the keyword
\sqsf{ch} and\sqsf{opn}, using the code database and HMM.
It can be very useful as a shorthand for handling long names,
but users should have a vague knowledge about the correct names
(e.g. \sqtt{chooser}, \sqtt{dialog}),
and large code example corpus should exist before

% [HMM]を使って正しいコードを予測して提示する
% `ch opn`みたいなテキトーな入力から`chooser.showOpenDialog()`みたいな正しいコードが候補に出る
% コードデータベースを解析
% [[コメント]]
% [増井俊之.icon] 2018/3/23 18:53
% コードのほんの一部を指定すると正しく予測されるというのは面白い
% HMMってそういうのに使えるのね
% しかし[* 関数名とか全然覚えていないときは使えないだろう]
% [語彙問題]が解決できてない
% だとすると高速入力の役にはたつが、何もかもわからない人をサポートはできない
% 用例データが存在しない場合も使えない
% [IDE]でしか使えない
% [Cyrus Omar: Active code completion]で参照されている

\subsection{Using cloud data for development}

% 他人の情報利用する系
% コード共有、コンパイルエラー共有

Many systems support sharing users' development experiences on the web so that
they can be used on the developers' IDE.

Using BluePrint\cite{Brandt:2010:EPI:1753326.1753402}
as a plugin for Adobe Flex Builder,
users can search sample codes from the Web and use it immediately
as 

%   サンプルコードをすぐ参照して利用できる[IDE]

HelpMeOut\cite{Hartmann:2010:OPS:1753326.1753478} tells users
how to handle compile errors,
based on the people's experiences of handling compile errors.

%= 他人のヒストリを使ってコンパイルエラー対処法をサジェスト


% \subsection{Using user forums}
% 
% Qiita, StackOverflow
%

\subsection{Literate Programming}

Literate Programming\cite{Knuth}

参考文献にすることはないか

JavaDoc
Jupyter
Eve

\section{Evaluation}

\section{Conclusion}


% REFERENCES FORMAT
% References must be the same font size as other body text.
\bibliographystyle{SIGCHI-Reference-Format}
\bibliography{paper}

\end{document}
